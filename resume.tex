%-------------------------
% Resume in Latex
% Author : Adapted by Roy Finkelberg from Sourabh Bajaj
% License : MIT
%------------------------

% TODO consider lowering font size to 10
\documentclass[letterpaper,11pt]{article}

\usepackage{latexsym}
\usepackage[empty]{fullpage}
\usepackage{titlesec}
\usepackage{marvosym}
\usepackage[usenames,dvipsnames]{color}
\usepackage{verbatim}
\usepackage{enumitem}
\usepackage[pdftex]{hyperref}
\usepackage{fancyhdr}


\pagestyle{fancy}
\fancyhf{} % clear all header and footer fields
\fancyfoot{}
\renewcommand{\headrulewidth}{0pt}
\renewcommand{\footrulewidth}{0pt}

% Adjust margins
\addtolength{\oddsidemargin}{-0.375in}
\addtolength{\evensidemargin}{-0.375in}
\addtolength{\textwidth}{1in}
\addtolength{\topmargin}{-.5in}
\addtolength{\textheight}{1.0in}

\urlstyle{same}
\hypersetup{
  urlcolor={blue}
}

\raggedbottom
\raggedright
\setlength{\tabcolsep}{0in}

% Sections formatting
\titleformat{\section}{
  \vspace{-4pt}\scshape\raggedright\large
}{}{0em}{}[\color{black}\titlerule \vspace{-5pt}]

%-------------------------
% Custom commands
\newcommand{\resumeItem}[2]{
  \item\small{
    \textbf{#1}{: #2 \vspace{-2pt}}
  }
}
\newcommand{\resumePlainItem}[1]{
  \item\small{#1 \vspace{-2pt}}
}

\newcommand{\resumeSubheading}[4]{
  \vspace{-1pt}\item[]
    \begin{tabular*}{0.97\textwidth}{l@{\extracolsep{\fill}}r}
      \textbf{#1} & #2 \\
      \textit{\small#3} & \textsc{\small #4} \\ % TODO should this be it instead of sc?
    \end{tabular*}\vspace{-5pt}
}

\newcommand{\resumeSubItem}[2]{\resumeItem{#1}{#2}\vspace{-4pt}}

\renewcommand{\labelitemii}{$\circ$}


\newcommand{\resumeSubHeadingListStart}{\begin{itemize}[leftmargin=*]}
\newcommand{\resumeSubHeadingListEnd}{\end{itemize}}
\newcommand{\resumeItemListStart}{\begin{itemize}}
\newcommand{\resumeItemListEnd}{\end{itemize}\vspace{-5pt}}

\newenvironment{resumeItemList}{\resumeItemListStart}{\resumeItemListEnd}
\newenvironment{resumeSubheadingList}{\resumeSubHeadingListStart}{\resumeSubHeadingListEnd}

%-------------------------------------------
%%%%%%  CV STARTS HERE  %%%%%%%%%%%%%%%%%%%%%%%%%%%%


\begin{document}

%----------HEADING-----------------
\begin{tabular*}{\textwidth}{l@{\extracolsep{\fill}}r}
  \textbf{\href{}{\huge Roy Finkelberg}} & Email: \href{mailto:roy@gatech.edu}{roy@gatech.edu}\\
  \vspace{-2.0pt}
  \href{}{} & Github: \href{https://github.com/rfinkelberg}{RFinkelberg} \\
\end{tabular*}


%-----------EDUCATION-----------------
\section{Education}
  \begin{resumeSubheadingList}
    \resumeSubheading
      {Georgia Institute of Technology}{Atlanta, GA}
      {Joint Bachelor and Master of Science in Computer Science}{}
      \begin{resumeItemList}
        \resumeItem{B.S in Computer Science (Aug. 2016 - Dec. 2019)}
        {GPA: 3.82/4.0; Major GPA: 3.93/4.0 \\ Concentrations: Artificial Intelligence, Embedded Computing}
        \resumeItem{M.S in Computer Science (Jan. 2020 - Dec. 2021)}
        {Specialization: Machine Learning}
      \end{resumeItemList}
  \end{resumeSubheadingList}


%-----------EXPERIENCE-----------------
\section{Experience}
  \begin{resumeSubheadingList}
    \resumeSubheading
      {NVIDIA}{Austin, TX}
      {Data Science - AI Infrastructure Intern (RAPIDS AI)}{May 2019 - Aug. 2019}
      \begin{resumeItemList}
      % TODO consider not having the bold parts
        \resumeItem{Applied Research}{Adapted deep representation learning methods (graph autoencoders) to develop scalable
        network analysis and link prediction methods for large cybersecurity networks}
        \resumeItem{cuML}{Developed pure GPU implementations for ordinal feature encoding and data train/test split modules,
        giving up to 290x speedups over CPU implementations on large ($\sim 10^7$ row) datasets}
        \resumeItem{cuDF}{Constructed, profiled, and optimized fundamental data science primitives such as one-hot encoding and scalar-vector binary operations
        in Cython and Numba, improving performance by 1.5x on wide datasets}
      \end{resumeItemList}
    % \resumeSubheading
    %   {Google}{Mountain View, CA}
    %   {Software Engineer}{Oct 2016 - Present}
    %   \resumeItemListStart
    %     \resumeItem{Tensorflow}
    %       {TensorFlow is an open source software library for numerical computation using data flow graphs; primarily used for training deep learning models.}
    %     \resumeItem{Apache Beam}
    %       {Apache Beam is a unified model for defining both batch and streaming data-parallel processing pipelines, as well as a set of language-specific SDKs for constructing pipelines and runners.}
    %   \resumeItemListEnd


    \resumeSubheading
    {Pindrop Security}{Atlanta, GA}
    {Software Engineering Intern - Research}{May 2018 - Aug. 2018}
    \begin{resumeItemList}
      \resumeItem{Test Engineering}{Developed an automated testing harness for a cloud based machine learning platform}
      \resumeItem{Scalable System Design}{Designed an abstract schema to streamline creation and integration of new models}
      \resumeItem{Health Monitoring}{Created a standardized interface for reporting and viewing model performance metrics through
      Datadog, reducing manual Research Scientist intervention by $\sim 70$ hours per week}
      \resumeItem{Model Optimization}{Scaled Scikit-Learn's DBSCAN algorithm to $\sim 10^6$ dimensional feature vectors using 
      Spotify's open source Annoy library}
    \end{resumeItemList}
  \end{resumeSubheadingList}

%-----------PROJECTS-----------------
% Projects that didnt make the cut
%< healthcare quality analytics and improvement recommendations
%< stance
%< mobile TPU hacking
% TODO maybe instead of bullets just list projects with short blurbs. can fit more that way
\section{Projects and Publications}
  \begin{resumeSubheadingList}
    \resumeSubheading
    {Piazza Automated Related Question Recommender}{Published: \textit{ACM Learning @ Scale 2019}}
    {Georgia Tech Contextual Computing Group}{Aug. 2018 - Present}
    \begin{resumeItemList}
      \resumePlainItem{Developed the Flask backend of a question recommendation engine which leverages the collective
      memory of classes with online forums to prevent duplicate posts}
      \resumePlainItem{Conducted A/B testing of model performance and impact across $1000+$ users}
      \resumePlainItem{Published as \textit{PARQR: Augmenting the Piazza Online Forum to Better Support Degree Seeking Online Masters Students},
      showing a 40\% reduction in duplicate posts} % TODO consider removing name here
    \end{resumeItemList}
    % As online classes grow in popularity, instructors turn to online forums to facilitate the discussion found 
    % in regular classrooms. While effective, these scale poorly. Increasing class sizes lead to an increased
    % number of duplicate posts and decreased situational awareness. PARQR is a recommendation engine which suggests
    % users existing posts to prevent 

    % TODO double dipping may be bad here
    \resumeSubheading
    {Towards Scalable Cybersecurity Network Analysis with Graph Autoencoders}{}
    {NVIDIA - RAPIDS AI}{Aug. 2019}
    \begin{resumeItemList}
      \resumePlainItem{Investigated the use of autoencoder based methods for large scale cybersecurity network analysis}
      \resumePlainItem{Adapted existing graph autoencoder architectures in Tensorflow and PyTorch to static and dynamic
      cybersecurity networks}
      \resumePlainItem{Published as an internal NVIDIA whitepaper, demonstrating up to 4x performance increases and 
      9x speedups on link prediction tasks}
    \end{resumeItemList}

    \resumeSubheading
    {Model Based Intention Detection for Intelligent Prostheses}{}
    {Georgia Tech Exoskeleton Prosthetic and Intelligent Controls Lab}{Aug. 2017 - Dec. 2017}
    \begin{resumeItemList}
      \resumePlainItem{Collected and analyzed biometric sensor data to determine features important to gait
      speed detection}
      \resumePlainItem{Preprocessed data and engineered features using Python's Scikit-Learn library} % TODO dont like the way this sounds
      \resumePlainItem{Presented a preliminary offline gait speed detection model demonstrating the effectiveness of these features}
      \resumePlainItem{Assisted in the design of a Kivy GUI which interfaced with ROS to visualize and adjust a prosthetic's control
      parameters during operation}
    \end{resumeItemList}
  \end{resumeSubheadingList}

%<----------SKILLS-------------------
\section{Skills}
  \begin{resumeItemList}
    \resumeItem{Languages}{Python (advanced), Java (intermediate), C (intermediate)}
    \resumeItem{Tools/Technologies}{Numpy/Scipy, PyTorch, Pandas, Numba (with CUDA), Cython, Pytest, \LaTeX,
                                    Software Integration, Hardware Prototyping, Linux Environments} % TODO experiemntal design/stat analysis?
  \end{resumeItemList}

%
%--------PROGRAMMING SKILLS------------
%\section{Programming Skills}
%  \resumeSubHeadingListStart
%    \item{
%      \textbf{Languages}{: Scala, Python, Javascript, C++, SQL, Java}
%      \hfill
%      \textbf{Technologies}{: AWS, Play, React, Kafka, GCE}
%    }
%  \resumeSubHeadingListEnd


%-------------------------------------------
\end{document}
